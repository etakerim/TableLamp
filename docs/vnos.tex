\documentclass[12pt, a4paper]{article}

\usepackage[slovak]{babel}
\usepackage[utf8]{inputenc}
\usepackage[T1]{fontenc}
\usepackage{geometry}
\usepackage{hyperref}
\usepackage{setspace}
\usepackage{subcaption}
\usepackage{afterpage}
\usepackage{graphicx}
\usepackage{csquotes}
\usepackage{longtable}
\usepackage{pdfpages}
\usepackage{expl3}

% Dots in TOC
\usepackage{tocloft}
\renewcommand{\cftsecleader}{\cftdotfill{\cftdotsep}}


\setstretch{1.5}
\widowpenalty10000
\clubpenalty10000
\newsavebox\shield
\usepackage{titlesec}

\usepackage[style=iso-numeric,backend=biber]{biblatex}
\addbibresource{literature.bib}
%\AtBeginBibliography{\small}

\geometry{
	a4paper,
	top=2cm,
	left=3cm,
	right=2.5cm,
	bottom=2.5cm
}

\begin{document}
\begin{titlepage}
{\centering
    {\Large Slovenská technická univerzita v Bratislave}\par
    {\Large Fakulta informatiky a informačných technológií}\par
    \vspace{\medskipamount}
    \vfill
    \LARGE \textbf{Inteligentné osvetlenie pracovného stola} \\
    \vspace{0.7\bigskipamount}
    {\Large Semestrálny projekt}\par
    \vfill
}
\normalsize    
\begin{flushleft}
\textbf{Autor:} Bc. Miroslav Hájek \\
\textbf{Študijný program:} Inteligentné softvérové systémy \\
\textbf{Predmet:} Vnorené systémy \\
\textbf{Akademický rok:} 2022 / 2023 \\
\end{flushleft}
\end{titlepage}

\thispagestyle{empty}
\includepdf[pages=-, scale=1]{zadanie}

\pagenumbering{gobble}
\tableofcontents
\newpage

\pagenumbering{arabic}
\setcounter{page}{1}

\section{Analýza problematiky}
Zariadenie na ovládanie svietidla sa má skladať z dvoch nezávislých integrovaných častí: riadiaca jednotka s firmvérom priamo regulujúca spínanie lampy podľa senzorov, a mobilná aplikácia na zmenu aktívnych nastavení svetla a detekcie pohybu komunikujúca cez Bluetooth. V sekcii popíšeme výber vhodných súčiastok a základné princíp fungovania.

\subsection{Riadiaca jednotka}
Mikrokontrolér \textbf{ESP32 WeMos Lolin D32} (Obr.~\ref{fig:esp32}) sme zvolili spomedzi ostatných bežne dostupných vývojových dosiek (ako sú Arduino, Raspeberry Pi, STM32, BeagleBone a pod.) hlavne pre podporu pre viacerých štandardných mikroprocesorových rozhraní na rozličných pinoch. Doska je postavená na module \emph{ESP32-WROOM-32} \cite{noauthor_esp32-wroom-32_2023}, ktorý disponuje procesorom s frekvenciou 80 MHz (až do 240 MHz) a pamäťou 4 MB Flash a 520 kB SRAM prilíš nelimitujúce rozhodnutia na účely prototypu. 

Napájacie napätie dosky je 3,3 V, ale cez USB konektor môžeme pripojiť zdroj 5 V. ESP-32 sa vyznačuje aj dobrou podporou od výrobcu v podobe SDK obsahujúce ovládače na všetky dostupné rozhrania pre periférie ESP-IDF \cite{noauthor_esp-idf_nodate}, ktorý tiež zahŕňa operačný systém reálneho času FreeRTOS.

\begin{figure}[h]
	\centering
	\includegraphics[width=\textwidth]{assets/esp32.jpg}
	\caption{Vývojová doska ESP32 WeMos Lolin D32 \cite{mischianti_esp32_2023}}
	\label{fig:esp32}
\end{figure}

ESP-32 síce disponuje konektivitou na WiFi a Bluetooth, ale funkcie pre RFCOMM nie sú zatiaľ implementované, preto sa Bluetooth komunikácia musí realizovať externým \textbf{modulom HC-05} (Obr.~\ref{fig:hc-05}). Posielanie a príjem správ sa uskutočnuje cez \emph{UART} protokol, kde piny RXD a TXD musia byť na napäťových úroviach 3,3 V, zatiaľ čo napájanie je v rozmedzí 3,6 - 6 V (zväčša 5 V). Baudová rýchlosť je predvolene určená na 9600. 

Pin Enable slúži na prepnutie modulu do AT príkazového režimu (high úroveň), v ktorom je možné upraviť napr. baud. Pin State signalizuje stav Bluetooth pripojenia a je zároveň zapojený na indikačnú LED, ktorá sa môže nachádzať v 3 stavoch: (i) bliknutie raz za 2 sekundy - príkazový režim, (ii) bliknutie 2-krát za 1 sekundu - pripojenie nadviazané, (iii) opakované blikanie - čakanie na pripojenie \cite{noauthor_hc-05_nodate}.

\begin{figure}[h]
	\centering
	\includegraphics[width=0.5\textwidth]{assets/hc-05.png}
	\caption{Bluetooth modul HC-05 \cite{noauthor_hc-05_nodate}}
	\label{fig:hc-05}
\end{figure}

\subsection{Senzory a akčné členy}
Regulovateľné LED diódy v úlohe akčného členu musia reagovať na situáciu v okolitom prostredí, menovite detekciu pohybu človeka v miestnosti (cez infračervené čidlo) a intenzitu vonkajšieho osvetlenia (cez fotodiódu). 

\subsubsection{RGB LED pásik}
Typ LED pásiku sa vyberá podľa rôznych kritérii ako sú účel použitia, či ide o dekoráciu alebo má slúžiť na osvetlenie miestnosti. Na základe toho prispôsobíme svetelný výkon, farbu, vodovzornosť (stupeň ochrany krytom), rozmery, z čoho nakoniec vyplýva cieľové napätie najčastejšie 12~V, 24~V, 230~V. 

Rozlišujeme pásiky podľa technológie produkcie teplotného odtieňu: jednofarebné, CCT - s dvoma typmi diód 2700~K a 6500~K a reguláciou medzi nimi, RGB - s štvorpinové LED diódami. Svietivosť pásiku ovplyvňuje príkon udávaný na meter, na nasvietenie napr. kuchynskej linky sa odporúča LED pásik s vyšším príkonom 12 - 20 W/m, tie je žiaduce namontovať na hliníkový podklad, z dôvodu chladenia. Ďalší faktor určujúci charakter svetla je rozloženie diód a ich počet na meter pásika \cite{123ledsk_ako_nodate}.

\begin{figure}[h]
\centering
\begin{subfigure}[b]{0.45\textwidth}
	\centering
	\includegraphics[width=\textwidth]{assets/rgb-led.jpg}
	\caption{LED pásik}
	\label{fig:rgb-led}
\end{subfigure}
\hfill
\begin{subfigure}[b]{0.5\textwidth}
	\centering
	\includegraphics[width=\textwidth]{assets/bd711.png}
	\caption{NPN tranzistor BD711}
	\label{fig:bd-711}
\end{subfigure}
\caption{Akčné členy svietidla}
\end{figure}

Vhodný kandidát na interiérové osvetlenie stola sa ukazuje \textbf{RGB LED pásik} (Obr.~\ref{fig:rgb-led}) so samolepiacou fóliou 3M 300LSE bez krytia IP20, napätím zdroja 12~V, príkonom 14,4~W/m a hustotou diód 60~ks/m, ktorý vyprodukuje až 550~lm/m svetla \cite{123ledsk_rgb_nodate}. LED diódy na pásiku sú zapojenie tri do série so spoločnou anódou v 10 cm oddeliteľných úsekoch. Vetva pre červenú farbu má navyše zapojený do série 300~$\Omega$ SMD rezistor, ostatné vetvy 150~$\Omega$ rezistor.

Jednotlivé farebných kanálov nemôžeme spínať a regulať priamo s pinu mikroprocesoru, pretože nie je prítomné rovnaké napätie: 3,3~V voči 12~V, a nepostačuje ani maximálny prúd $I_{OL}$ = 28 mA. Dedikovaný tranzistor bude spínať prúd do 1,2 A ($I = 14,4W / 12V$), preto potrebujeme výkonovejšie tranzistory ako bežne používaný rad BC ($I_C < 100 mA$). \textbf{NPN tranzistor BD711} (Obr.~\ref{fig:bd-711}) dokáže spínať prúd až do 12 A \cite{noauthor_bd711_nodate}.

\begin{figure}[h]
\centering
\begin{subfigure}[b]{0.4\textwidth}
	\centering
	\includegraphics[width=\textwidth]{assets/pir-sb312.jpg}
	\caption{PIR senzor pohybu SB312}
	\label{fig:pir}
\end{subfigure}
\hfill
\begin{subfigure}[b]{0.4\textwidth}
	\centering
	\includegraphics[width=\textwidth]{assets/tsl25911.jpg}
	\caption{Senzor osvetlenia TSL25911}
	\label{fig:light-sensor}
\end{subfigure}
\caption{Senzory s označením vývodov}
\end{figure}

\subsubsection{PIR pohybový senzor}
Na detekciu pohybu ľudí sa používajú infračervené PIR senzory líšiace sa veľkosťou, citlivosťou a nastaviteľnosťou úrovne detektora. Zlepšenie snímacích vlastností je docielené plastovou Fresnelovou šošovkou na senzore. \textbf{Olimex SB312} (Obr.~\ref{fig:pir})sa vyzančuje kompaktnosťou s rozmermi PCB: 10 x 8 mm. Disponuje uhlom snímacieho kužela do 100° s dosahom 3 - 5 metrov. Oneskorenie výstupu voči detegovanému pohybu je 2 sekundy a po rovank dlhý čas podrží výstupnú logickú úroveň v jednotke. Napájacie napätie senzora je 3,3~V \cite{olimex_pir-sb312_nodate}.

\subsubsection{Senzor okolitého osvetlenia}
Modul senzora osvetlenia Waveshare WS-17146 (Obr.~\ref{fig:light-sensor}) obsahuje obvod TSL25911FN, ktorý samostatne prerátava nameranú intenzitu svetla na fotodióde do luxov s použím vzorca na aproximáciu ľudského videnia. Senzor pracuje na napätí 3,3~V aj 5~V a komunikuje cez I2C zbernicu do 400 kbit/s na adrese 0x29. Najdôležitejšie registre pre odčítanie aktuálnej hodnoty sú Control (0x01) a ALS Data (0x14 - 0x17) \cite{noauthor_tsl25911_nodate}.

\subsection{Prehľad hardvérových súčiastok}
Na základe predošlej analýzy sme zostavili zoznam súčiastok potrebných pre zhotovenie zariadenia. Cena hardvéru prototypu (jednokusová výroba) je odhadovaná na 55 € na základe nami zvolených dodávateľov.

\paragraph{Riadiaca jednotka}
\begin{itemize}
\item ESP32 WeMos Lolin D32 - Mikrokontrolér
\item HC-05 - Bluetooth modul
\end{itemize}

\paragraph{Senzory a akčné členy}
\begin{itemize}
\itemsep0pt
\item RGB LED pásik 14,4W/m 12V bez krytia IP20 1m
\item Olimex PIR-SB312 10x8mm 
\item WS-17146 TSL25911 Light Sensor
\end{itemize}

\paragraph{Elektrické súčiastky}
\begin{itemize}
\itemsep0pt
\item L7805CV Lineárny regulátor napätia 12V na 5V
\item NPN Tranzistor BD711 
\item Rezistory - 220 $\Omega$, 330 $\Omega$
\item LED zdroj (trafo) 12V 30W IP67
\item Flexo šnúra – 3m
\item Konektory: RGB LED pásik, Micro USB-B, DC konektor a zásuvka
\end{itemize}

\paragraph{Mechanické súčiastky}
\begin{itemize}
\itemsep0pt
\item DO1 chladič
\item DO3A chladič
\item Univerzálny plošný spoj
\item Nástenný profil N3 biely, Opálový kryt 1m
\item Koncovka profilu N3 biela
\item Vypínač mezišnúrový
\end{itemize}

\section{Návrh riešenia}
\begin{figure}
	\centering
	\includegraphics[width=\textwidth]{assets/block-diagram.png}
	\caption{Blokový diagram vnoreného systému}
\end{figure}

\begin{figure}
	\centering
	\includegraphics[width=0.8\textwidth]{assets/electrical-schematics.png}
	\caption{Schéma obvodu}
\end{figure}

\subsection{Low-fidelity prototyp aplikácie}
\begin{figure}
	\centering
	\includegraphics[width=0.4\textwidth]{assets/wireframe.png}
	\caption{Wireframe mobilnej aplikácie}
\end{figure}

\subsection{Spínanie svetla}
\begin{figure}
    \centering
	\includegraphics[width=0.7\textwidth]{assets/light-states.png}
	\caption{Stavový diagram spínania svietidla}
\end{figure}

\begin{enumerate}
\item Kelvin ← Temperature
\item Kelvin →RGB (podľa regresného modelu)
\item RGB → HSV
\item V ←brightness
\item HSV→ RGB
\item PWM driver 8-bit OUT
\end{enumerate}

\newpage
\section{Opis implementácie}
% Firmvér v ESP-IDF
% Softvér pre Android smartfón

\begin{figure}
\includegraphics[width=\textwidth]{assets/prototype.jpg}
\end{figure}

\begin{figure}
\includegraphics[width=\textwidth]{assets/pir-motion-detect.png}
\end{figure}

\subsection{Nastavenie farby a intenzity svietidla}

\subsection{Firmvér tasky}

% ESP-IDF - C kód a drivers 
% Android studio  (link na dokumentáciu)

\subsection{Mobilná aplikácia}
Obrazovka a popis fungovania

\section{Otestovanie riešenia}
% Testovacie scenáre - Akceptačné testy

% - 

\section{Zhodnotenie výsledkov}

\printbibliography[title={Literatúra}]
\newpage

\section{Technická dokumentácia}
% Štruktúra priečinkov
% Nahratie firmvéru
% Nahratie softvéru
% FW a SW: Moduly a funkcie

\end{document}